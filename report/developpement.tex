Les clients et le serveur communiquent en s'envoyant divers messages. Ces messages concernent l'état des joueurs (position, points de vie, nombre de bombes, \dots) ou des informations sur la partie (démarrage, début du tour d'un joueur, \dots). Ces messages prennent la forme de tableaux de caractères et sont interprétés par le client ou le serveur.

\section*{Serveur}
\addcontentsline{toc}{section}{\protect\numberline{}Serveur}

\subsection*{Fonctionnement général}
\addcontentsline{toc}{subsection}{\protect\numberline{}Fonctionnement général}

L'exécution du serveur prend deux paramètres, le nombre de parties simultanées maximum autorisé (entre 1 et 100), et le nombre de joueurs par partie (entre 2 et 4).

Une fois lancé, il réserve la place mémoire nécessaire pour le nombre de parties demandé, puis écoute les demandes de connexion des clients sur le port 5000.\\

Lors de la réception d'un message sur ce port, il cherche à déterminer si une partie est capable d'accueillir un joueur supplémentaire.

Pour ce faire, il parcourt la liste des parties, à la recherche d'une n'ayant pas assez de joueurs pour démarrer. Dès qu'il trouve une partie dans cet état, il arrête la recherche et ajoute un joueur à la partie. Il envoie ensuite au client un message contenant le numéro de la partie en question et le numéro de joueur du client au sein de cette partie.

Si aucune partie ne manque de joueurs, et que le nombre maximal de parties n'est pas atteint, le serveur crée une nouvelle partie. Il y ajoute ensuite un joueur, et envoie au client un message de la forme décrite précédemment.

Enfin, si aucune partie ne manque de joueurs et que le nombre maximal de parties simultanées est atteint, le serveur envoie au client un message indiquant qu'aucune partie n'est disponible.

\subsection*{Gestion d'une partie}
\addcontentsline{toc}{subsection}{\protect\numberline{}Gestion d'une partie}

Afin de ne pas créer de conflit et pour éviter les blocages ou arrêts du serveur, chaque partie est gérée dasn un thread qui lui est propre. Ainsi, dans le cas d'un problème tel que le blocage de la partie, le serveur est toujours en mesure de gérer l'arrivée de nouveaux clients.

La mise en place d'un tel mécanisme a cependant nécessité de pouvoir contrôler les accès concurrents aux parties lors du parcours visant à déterminer si un nouveau joueur peut être accepté ou non. Pour ce faire, nous avons utilisé les mutex tels qu'implémentés par la \texttt{CSFML}.\\

\`A sa création, une partie écoute les messages envoyés par ses joueurs sur un port qui lui est propre. Le numéro de ce port est déterminé selon la formule suivante: \(5100+numeroPartie*(joueursParPartie+1)\), ce qui donne par exemple 5100, 5105 et 5110 pour 3 parties de 4 joueurs.

Une fois tous les joueurs présents, un message leur est envoyé pour indiquer le début de la partie, et le premier joueur reçoit également le message indiquant qu'il doit effectuer une action.\\

La partie exécute alors la boucle suivante, jusqu'à ce qu'il ne reste qu'un seul joueur en vie:
\begin{itemize}
	\item Attente de l'action du client du joueur courant.
    \item Vérification de la validité de l'action du client. Si celle ci n'est pas valide, un message lui est envoyé demandant de refaire une action. Si celle ci est valide, on passe à l'étape suivante.
    \item Réalisation de l'action (déplacement ou pose d'une bombe).
    \item Avancement de l'état des bombes posées par le joueur courant. Si l'une d'elles explose, les dégâts infligés sont calculés.
    \item Envoi à tous les joueurs du nouvel état du jeu.
    \item Indication au joueur suivant que c'est à lui de jouer.\\
\end{itemize}

Une fois la partie finie, un message est envoyé aux joueurs pour le leur indiquer. La partie est ensuite réinitialisée, et prête à accueillir de nouveaux joueurs.

\section*{Client}
\addcontentsline{toc}{section}{\protect\numberline{}Client}

Avant de lancer les divers clients, il faut s'assurer que le serveur tourne déjà. Lorsque un client est exécuté, il doit avoir en paramètres : une adresse (l'adresse ip du serveur) et une chaîne de caractères (le nom du joueur). Grâce à ces paramètres, le client demande une connexion au serveur qui, s'il n'est pas surchargé, répond sur le port 5001 en lui renvoyant le numéro de la partie qui lui a été attribuée et son numéro de joueur. Si le serveur est surchargé, il répond qu'aucune partie n'est disponible.


\subsection*{Déroulement d'une partie}
\addcontentsline{toc}{subsection}{\protect\numberline{}Déroulement d'une partie}

Une fois la connexion acceptée et le client ajouté à une partie, les messages du serveur sont reçus sur un port dépendant du numéro de partie et du numéro du joueur dans cette partie, calculé de la façon suivante: \( 5100 + numeroPartie * (joueursMaxParPartie + 1) + numeroDeJoueur\). Ceci permet de lancer autant de client que possibles sur une seule machine sans avoir pour autant de problème pour envoyer le bon message au bon client.

Lorsque le client est connecté au serveur et qu'il possède un numéro de partie et un numéro de joueur, alors la partie peut commencer. Le joueur dispose de deux états, mis à jour par le serveur : l'un lorsque c'est à son tour de jouer, et l'autre quand ça ne l'est pas.\\ 

\underline{Si c'est au tour du client de jouer}, alors celui-ci récupère une action de l'utilisateur et l'envoie au serveur. Le joueur ne peut faire que deux actions différentes, et une seule par tour : se déplacer ou poser une bombe. Lorsque le serveur à reçu l'action du joueur par le client, il doit alors lui renvoyer un message qui peut être :
\begin{itemize}
	\item Soit un message disant que l'action de l'utilisateur est réfusée (déplacement incorrect ou impossibilité de poser une bombe). Dans ce cas le client récupère une nouvelle action de l'utilisateur, la renvoie au serveur et re-attend un nouveau message du serveur.
	\item Soit un message avec les nouvelles coordonnées des joueurs et des bombes, qui signifie que l'action à été acceptée, que le client doit mettre à jour l'interface graphique, et que ce n'est plus à son tour de jouer.
\end{itemize}

\vspace{0.5cm}

\underline{Si ce n'est pas au tour du client de jouer}, alors celui-ci peut recevoir deux messages différents du serveur :
\begin{itemize}
	\item Soit un message lui annonçant que tous les joueurs ont fini de jouer et que c'est à son tour.
	\item Soit un message avec les nouvelles coordonnées des joueurs et des bombes pour que le client mette à jour l'interface graphique.\\
\end{itemize}

Ces étapes sont répétées en boucle jusqu'au message du serveur indiquant que la partie est terminée.

Le client détermine alors, selon ses points de vie restants, s'il a gagné ou non.

\subsection*{Mise à jour graphique et fin de partie}
\addcontentsline{toc}{subsection}{\protect\numberline{}Mise à jour graphique et fin de partie}

A chaque fois que le client reçoit un message du serveur contenant des informations sur les autres joueurs, il met à jour l'interface graphique. Pour cela il fait d'abord des tests sur les données des autres joueurs afin de déterminer :
\begin{itemize}
	\item la position actuelle de tous les joueurs.
	\item la direction du regard des joueurs en fonction de leur mouvement.
	\item l'emplacement des bombes.
	\item l'état des bombes : juste posée, en décompte ou en train d'exploser.
	\item les points de vie restant aux joueurs et les joueurs mort.
	\item le statut de la partie : si il y a un vainqueur ou non.
\end{itemize}

\vspace{0.5cm}

Ensuite, lorsque toutes les données ont été analysée, le client choisit les sprites correspondantes et les places dans la fenêtre. Cette dernière est ensuite actualisée et affichée sur l'écran du joueur qui a exécuté le client. 
