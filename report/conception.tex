\section*{Dépendances externes}
\addcontentsline{toc}{section}{\protect\numberline{}Dépendances externes}

	Avant même de concevoir un logiciel, il faut au préalable le penser, ainsi que définir ses besoins et ses diférentes fonctionnalités. Dans cette optique, notre cahier des charges nous imposait de concevoir l'application en \texttt{langage C}. Ce langage ne fût pas choisi pour ses performances dans la conception d'applications réseau, mais plutôt dans une optique pédagogique. En effet, il est moins évident de programmer des échanges entre client et serveur en \texttt{langage C} plutôt qu'en \texttt{Java} ou en \texttt{C++} où tout est mis à disposition pour aider le programmeur.

\vspace{0.5cm}

	En ce qui concerne la partie graphique de l'application, nous avons choisi d'utiliser le \texttt{CSFML} qui est le binding officiel de SFML pour le langage C. La \texttt{SFML} étant une Interface de programmation destinée à construire des jeux vidéo ou des programmes interactifs en \texttt{C++}.

\section*{Fonctionnalités du logiciel}
\addcontentsline{toc}{section}{\protect\numberline{}Fonctionnalités du logiciel}

	La seule fonctionnalité qui nous a été imposée pour ce projet est la conception d'un serveur pour le jeu capable d'échanger divers messages avec plusieurs clients joueurs. Ainsi, nous avons mis en place un serveur capable de gérer au maximu XXX parties comprenant chacune de 2 à 4 joueurs. 

			principe du jeux / échanges de messages

\section*{Interface graphique}
\addcontentsline{toc}{section}{\protect\numberline{}Interface graphique}
			image / description

