Nous avons présenté, tout au long de ce rapport, notre application \texttt{Bomber Unicorn}. Malgré la simplicité des règles de ce jeu, le projet fut relativement difficile à réaliser. En effet, nous avons utilisé les nouvelles notions de communications réseaux étudiées en cours, dans un langage de programmation bas niveau, ce qui complexifie les choses. De plus, l'utilisation de plusieurs processus (multithreading) sur des ordinateurs peu puissants nous a par moment causé des problèmes de performances et d'efficacité.\\

 Malgré ces difficultés, nous avons réussi à mener le projet dans un état fonctionnel, mais il reste cependant quelques points que nous aurions souhaité améliorer, si nous en avions eu le temps. En effet, l'ajout d'informations dans l'interface graphique du client serait un gain de clarté pour l'utilisateur (description des tours de jeu, nombre de points de vie restants, etc...).
 
 De plus, certains problèmes restent à corriger, comme l'absence de timeout pour gérer le cas où un client ne répondrait plus en cours de jeu (ce qui bloque actuellement une partie). Des erreurs de segmentation surviennent également aléatoirement côté serveur lors de la connexion du premier client au serveur, probablement dûes au lancement d'un thread. Ces erreurs semblent cependant être beaucoup moins présentes si le serveur est exécuté via un débuger tel que gdb.\\

\texttt{Bomber Unicorn} nous a permis d'utiliser de nouvelles connaissances étudiées en \textit{Réseaux et protocoles Internet}, mais pas seulement. Nous nous sommes servis pour ce projet d'autres notions vues dans d'autres modules d'enseignement.