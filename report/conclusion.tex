Nous avons présenté, tout au long de ce rapport, notre application \texttt{Bomber Unicorn}. Malgré la simplicité de ce jeu, le projet fut relativement difficile à réaliser. En effet, nous avons utiliser les nouvelles notions de communications réseaux étudiées en cours, dans un langage de programmation plutôt limité, et inadapté à la mise en place d'une interface graphique. De plus, l'utilisation de plusieurs procéssus (multithreading) sur des ordinateurs peu puissant nous a grandement handicapé pour la gestion conception de l'application.\\

 Malgré ces difficultés, nous avons réussi à finaliser le projet, mais il reste cependant quelques points que nous aurions souhaité améliorer, si nous en avions eu le temps. En effet, après avoir fait tester notre jeu à plusieurs personnes, nous nous sommes rendu compte qu'il aurait pu être utile d'ajouter du texte sur l'interface graphique (description des tours de jeu, nombre de points de vie restant, etc...).\\

\texttt{Bomber Unicorn} nous a permis d'utiliser de nouvelles connaissances étudiées en \textit{Réseaux et protocoles Internet}, mais pas seulement. Nous nous sommes servis pour ce projet d'autres notions vues dans d'autres modules d'enseignement.
